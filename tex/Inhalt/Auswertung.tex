\chapter{Fazit und Ausblick}

% Blick in Bachelorarbeit

% Zusammenfassung
- Ziel war es ...

% Theoretische Grundlagen
- Nach der Diskussion verwandter Arbeiten in Kapitel x wurden zun�chst die f�r die Arbeit relevanten graphentheoretischen Begriffe definiert.
- graphentheoretische Grundlagen (Definitionen)
- Anwendungszenarien / Arten von Netzwerken -> Schwerpunkt auf Informations- und Wissensnetzwerke
- Die Betrachtung verschiedener Netzwerkarten hat verdeutlicht, dass es graphenbasierte Softwaresysteme mit verschiedenen Schwerpunkten geben muss ...

- Top Down Prinzip!: graphenbasierte Softwaresysteme -> Graphdatenbanksystemen -> Vorauswahl -> funktionaler Vergleich -> Auswahl -> Benchmark

- Kategorisierung: Graphdatenbanksysteme, Graph Processing Systems, Analyse- und Visualisierungssoftware
	- weiter auf GDBMS eingegangen -> Definition und weitere Unterteilung
		- Def: OLTP-artig (Mehrbenutzerf�higkeit, transaktionale Performance, Integrit�t, Verf�gbarkeit)
			- nativ / nicht-nativ
			- zentral / verteilt
			- eingebettet / Client-Server
			- disk- / hauptspeicherzentriert
			
- Datenmodelle (betrachtet: PGM, PHGM, RDF)
	- PGM: gerichteter, kantenbezeichneter, attributierter Multigraph
		- flexibel -> f�r viele Anwendungsf�lle einsetzbar
		- Verantwortung bzgl. Schemaverwaltung an Anwendung �bergeben
		- semistrukturierte Daten (Schemaevolution) -> Integration heterogener Datenquellen
	- PHGM: Erweiterung um n-�re Beziehungen .. seltenes, sehr spezielles Datenmodell
	- RDF in Verbindung mit SPARQL:
		- Schwerpunkte Inferenz und virtuelles Integrieren entsprechen nicht den Schwerpunkten des Forschungsvorhabens
		- fehlende modellinh�rente Differenzierung in eine Beziehung zwischen Ressourcen und einer Beschreibung einzelner Ressourcen zwischen Attribut und Beziehung erschwert das Formulieren analytischer Anfragen
		- keine dynamischen Pfadinstanzen
	-> PGM und (P)HGM f�r das Forschungsvorhaben geeignet
	
- Operationen:
	- grundlegend: 
		- CRUD zur Definition, Manipulation und Auslesen der Datenbasis
		- Traversierung: Beschreiben abstrakter Wege, Grundlage f�r komplexere Operationen
	- komplexe Operationen:
		- Erreichbarkeit: Pfade beliebiger und fester L�nge, k�rzeste Pfade
		- Mustersuche: exakt / inexakt (Unterscheidung in Isomorphismus und Subgraphisomorphismus / Subgraphhomomorphismus)
		- Aggregation als nutzdatenbasiertes Zusammenfassen: insbesondere f�r die analytische Anwendung interessant
		- Summierung als topologisches Zusammenfassen
		- Metriken (zur Analyse des gesamten Graphen) werden in den verglichenen Systmen nicht direkt unterst�tzt und m�ssen bei Bedarf hinzugef�gt werden
		
- funktionaler Vergleich von GDBMS
	- Beschreibung des Forschungsvorhabens:
		- 
	- Vorauswahl 22 auf 4 Systeme
		- Betrachtung in verschiedenen Kategorien

% Funktionaler Vergleich

% Benchmark
- hat ergeben, dass

% Fazit
- Verbinden der Ergebnisse aus funktionaler Evaluation und Benchmark


% Ausblick

% generell
- standardisierte Anfragesprache fehlt weiterhin
	- Gremlin/Blueprints bildet Quasi-Standard
		- Anpassung der GDBMS-Wrapper ist Community-driven (verz�gert)
	- keine Bestrebungen hinsichtlich Standardisierung erkennbar
	- Cypher ist guter Kandidat f�r Standard
- Eignung Graph Processing f�r Forschungsvorhaben
	
- R�ckwirkung auf Dokumentationen (OrientDB Transaktionen, Titan Kardinalit�t. Neo4j Anfragebeispiele)


% Ergebnis / Fazit


- Neo4j
	- Verwendung von Cypher
		- bei Performance-Problemen (Erreichbarkeit)... Nutzung der nativen API
		- Einbettung von Cypher im Quellcode -> hoher Wartungsaufwand (Holzschuher)
		- einfaches, intuitives Formulieren graphenbasierter Anfragen
- Titan

- beide
	- analytische Anfragen m�glich (Cypher, Gremlin)
	- Gremlin
		- h�herer Aufwand bei der Anfrageformulierung



% Benchmark
- Import: lesen und Schreiben von einer Platte (unrealistisch)
- parallele Lesezugriffe
- warmup-Prozedur verhindert Feststellen der Leistungsf�higkeit des Speichersystems
	-> weiterf�hrende Benchmarks
- Betrachtung ohne Caches sinnvoll?
	
% Ausblick


- Neo4j:
	- In-Memory-Erweiterungen f�r Analyse
	- Query-Optimierung
	- formale Untersuchung und Standardisierung von Cypher im GDBMS-Sektor
	- ausf�hrlicher Benchmark (parallele Queries, reale Unternehmensdaten)
	
- OrientDB
	- neues Speichersystem evaluieren
- HyperGraphDB
	- Implementierung Blueprints API
	
- quelloffene hauptspeicher-zentrierte Implementierung



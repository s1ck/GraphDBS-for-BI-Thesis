\chapter{Fazit und Ausblick}

% Zusammenfassung
- Ziel war es ...


- keine nativen APIs
	- Performancevorteil wurde bereits in verwandten Arbeiten gezeigt
% Ergebnis / Fazit
- Verbinden der Ergebnisse aus funktionaler Evaluation und Benchmark

- Neo4j
	- Verwendung von Cypher
		- bei Performance-Problemen Nutzung der nativen API
		- Einbettung von Cypher im Quellcode -> hoher Wartungsaufwand (Holzschuher)
		- einfaches, intuitives Formulieren graphenbasierter Anfragen
- Titan

- beide
	- analytische Anfragen m�glich (Cypher, Gremlin)
	- Gremlin
		- h�herer Aufwand bei der Anfrageformulierung

% Ausblick

- Cypher als standardisierte Sprache (Holzschuher und Peinl) kann nur unterst�tzt werden
- Gremlin wird nicht prim�r von Neo Technologies, sondern von Aurelius entwickelt
	- Unterst�tzung aktueller Features ist Community-Arbeit

- R�ckwirkung auf Dokumentationen (OrientDB Transaktionen, Titan Kardinalit�t. Neo4j Anfragebeispiele)

% Benchmark
- Import: lesen und Schreiben von einer Platte (unrealistisch)
- parallele Lesezugriffe
- warmup-Prozedur verhindert Feststellen der Leistungsf�higkeit des Speichersystems
	-> weiterf�hrende Benchmarks
- Betrachtung ohne Caches sinnvoll?
	
% Ausblick
- standardisierte Anfragesprache fehlt weiterhin
	- Gremlin Quasi-Standard
	- keine Bestrebungen erkennbar

- In-Memory-Erweiterungen f�r Analyse
- Query-Optimierung f�r Neo4j
- formale Untersuchung und Standardisierung von Cypher im GDBMS-Sektor
- ausf�hrlicher Benchmark (parallele Queries, reale Unternehmensdaten)
	
- OrientDB neues Speichersystem effizienter?
- HyperGraphDB -> Implementierung Blueprints API
- hauptspeicher-zentrierte Implementierung


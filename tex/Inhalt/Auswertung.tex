\chapter{Fazit und Ausblick}

- OrientDB neues Speichersystem effizienter?
- HyperGraphDB -> Implementierung Blueprints API
- hauptspeicher-zentrierte Implementierung

- analytische Anfragen m�glich (Cypher, Gremlin)

- Cypher als standardisierte Sprache (Holzschuher und Peinl) kann nur unterst�tzt werden
- Gremlin wird nicht prim�r von Neo Technologies, sondern von Aurelius entwickelt
	- Unterst�tzung aktueller Features ist Community-Arbeit

- R�ckwirkung auf Dokumentationen (OrientDB Transaktionen, Titan Kardinalit�t. Neo4j Anfragebeispiele)


% Benchmark
- Import: lesen und Schreiben von einer Platte (unrealistisch)
- parallele Lesezugriffe
- warmup-Prozedur verhindert Feststellen der Leistungsf�higkeit des Speichersystems
	-> weiterf�hrende Benchmarks
	
% Ausblick
- In-Memory-Erweiterungen f�r Analyse
- Query-Optimierung f�r Neo4j
- formale Untersuchung und Standardisierung von Cypher im GDBMS-Sektor
- ausf�hrlicher Benchmark (parallele Queries, reale Unternehmensdaten)
	

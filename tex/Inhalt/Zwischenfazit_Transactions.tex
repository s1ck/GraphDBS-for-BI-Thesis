\paragraph*{Transaktionsverwaltung}

Bei der Speicherung von Unternehmensdaten besteht ein hoher Anspruch an die Einhaltung der Datenkonsistenz und die Toleranz gegen�ber Systemfehlern. Aus diesem Grund wurde die Einhaltung der ACID-Anforderungen analysiert, Tabelle \ref{tab:zusammenfassung_transaktion} fasst die Ergebnisse zusammen.

\renewcommand{\arraystretch}{1.25}
\begin{table}[h]
	\centering
	\begin{footnotesize}
   	\begin{tabular}{|m{3cm}|>{\centering}m{2.9cm}|>{\centering}m{2.9cm}|>{\centering}m{2.9cm}|>{\centering\arraybackslash}m{2.9cm}|}
	\hline	
   	 \textbf{GDBMS} & \textbf{Neo4j} & \textbf{HyperGraphDB (BerkeleyDB)} & \textbf{OrientDB} & \textbf{Titan (BerkeleyDB)} \\   
   	\hline
%	\multicolumn{5}{|c|}{\textbf{Transaktionen}} \\   	
%	\hline
	\textbf{ACID-Eigenschaften} & ACID & ACI(D) & ACI(D) & ACID \\
   	\textbf{Isolationsebene} & \texttt{READ COMMITTED} & \texttt{SERIALIZABLE} & \texttt{SERIALIZABLE} (embedded) & \texttt{REPEATABLE READ} \\
   	\textbf{Synchronisation\newline~(GDBMS)} & RX-Sperrverfahren & MVCC & MVCC & RX-Sperrverfahren \\
   	\textbf{Synchronisation\newline~(Storage)} & - & RX-Sperrverfahren & - & RX-Sperrverfahren \\
   	\textbf{Schachtelung} & \checkmark (kein isoliertes R�cksetzen) & \checkmark (isoliertes R�cksetzen) & - & \checkmark (keine Isolation) \\
   	\hline
	\end{tabular} 
	\end{footnotesize}
	\setlength{\belowcaptionskip}{0.25cm}	
	\caption[Zusammenfassung: Transaktionsverwaltung]{Gegen�berstellung von ACID-Eigenschaften und Transaktionsmechanismen der betrachteten GDBMS.}
	\label{tab:zusammenfassung_transaktion}
\end{table}
\renewcommand{\arraystretch}{1}

Es zeigt sich, dass Neo4j die ACID-Eigenschaften erf�llt, jedoch die im Vergleich geringste Isolationsebene aufweist. Sollen im parallelen Betrieb entsprechende Mehrbenutzeranomalien vermieden werden, so liegt dies analog zu Schemaverwaltung und Konsistenzerhaltung in der Verantwortung der Anwendung. Da jedoch das Forschungsvorhaben auf den analytischen und somit vorrangig lesenden Betrieb ausgerichtet ist, kann die geringe Isolationsebene toleriert werden.\\
HyperGraphDB nutzt BerkeleyDB f�r die Einhaltung der ACID-Eigenschaften. F�r eine bessere Performanz wird dabei auf eine garantierte Dauerhaftigkeit von �nderungen verzichtet, diese l�sst sich jedoch manuell aktivieren. F�r den isolierten Zugriff auf den Cache implementiert das GDBMS ein MVCC-Verfahren, welches die h�chste Isolationsebene garantiert. HyperGraphDB bietet als einziges der betrachteten Systeme das isolierte R�cksetzen geschachtelter Transaktionen an und eignet sich somit speziell f�r langlaufende �nderungstransaktionen, die auf mehrere untergeordnete Transaktionen aufgeteilt werden.\\
Ein Systemausfall kann auch in OrientDB zum Verlust der �nderungen erfolgreich beendeter Transaktionen f�hren. Um diesem Szenario zu begegnen, besteht die M�glichkeit, Wiederherstellungsinformationen unter Inkaufnahme eines Geschwindigkeitsverlustes auf den Externspeicher zu forcieren. OrientDB garantiert im eingebetteten Betrieb die h�chste Isolationsebene, dies gilt jedoch nicht f�r den entfernten Zugriff.\\
Die ACID-Eigenschaften von Titan variieren je nach eingesetztem Speichersystem, die Verwendung von BerkeleyDB erfolgt mit Standardeinstellungen, was folglich die Dauerhaftigkeit von Transaktionen garantiert. Das GDBMS selbst stellt die Einhaltung definierter Integrit�tsbedingungen, wie zum Beispiel \texttt{UNIQUE}-Bedingungen, durch das Setzen von Sperren sicher.

Die Verwendung eines pessimistischen Synchronisationsverfahrens f�hrt dazu, dass sich entsprechende Systeme vorrangig f�r kurze �nderungstransaktionen eignen, da parallel ausgef�hrte Transaktionen blockiert werden k�nnen. Im Forschungsvorhaben sollen haupts�chlich Lesetransaktionen f�r die lokale Analyse des Graphen durchgef�hrt werden, �nderungen hingegen erfolgen in periodischen Bulk-Load-Prozessen. Die Verwendung eines RX-Sperrverfahrens stellt somit im Vergleich kein Defizit dar. Zu beachten ist, dass alle GDBMS die Atomarit�t von �nderungen durch deren Zwischenspeicherung im Hauptspeicher sicherstellen, was bei langen Lese- oder �nderungstransaktionen zu Problemen f�hren kann.
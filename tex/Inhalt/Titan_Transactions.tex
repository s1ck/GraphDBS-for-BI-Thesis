\paragraph*{Transaktionen}

Lese- und Schreibzugriffe erfolgen in Titan grunds�tzlich im Kontext einer Transaktion, die beim ersten Zugriff automatisch gestartet wird und an den erzeugenden Thread gebunden ist. Alle nachfolgenden Operationen nutzen die aktive Transaktion bis diese explizit beendet wird, hierf�r stehen Funktionen f�r Commit und Rollback zur Verf�gung. �nderungen an der Datenbasis werden zun�chst ausschlie�lich isoliert im Kontext einer Transaktion ausgef�hrt. Wird die Transaktion zur�ckgesetzt, werden alle bisherigen �nderungen verworfen. Kommt es w�hrend der Transaktionsausf�hrung oder beim Beenden zu einem Fehler, wird dieser an die Anwendung weitergegeben. Titan unterscheidet zwischen tempor�ren und dauerhaften Fehlern: Ein tempor�rer Fehler zum Beispiel ist der kurzzeitige Verbindungsabbruch im Client-Server-Betrieb, das GDBMS bietet die Option, entsprechend abgebrochene Transaktionen automatisch zu wiederholen. Als dauerhafte Fehler klassifiziert Titan zum Beispiel System- oder Hardwarefehler, bei denen das GDBMS gestoppt wurde. 

Neben den thread-gebundenen unterst�tzt Titan auch thread-unabh�ngige Transaktionen. Diese m�ssen explizit gestartet werden und erm�glichen es, mehrere Threads innerhalb einer Transaktion auszuf�hren. Hierdurch l�sst sich zum Einen parallele Hardware in Graphalgorithmen effizienter nutzen, zum Anderen erm�glicht es die Schachtelung von Transaktionen. Dabei ist jedoch zu beachten, dass �nderungen untergeordneter Transaktionen nicht im Kontext der �bergeordneten Transaktion erfolgen und somit nach erfolgreichem Commit global sichtbar sind. Ein Zur�cksetzen der �bergeordneten Transaktion wird nicht rekursiv auf die eingebetteten Transaktionen angewendet.

Die Verantwortung f�r die Einhaltung der ACID-Eigenschaften innerhalb des Speichersystems bei der Ausf�hrung nebenl�ufiger Commits wird an die entsprechende Speicherschicht �bergeben. BerkeleyDB wird in Titan mit Standardeinstellungen verwendet, d.h. die Isolationsebene entspricht \texttt{REPEATABLE READ} und es werden somit alle Mehrbenutzeranomalien au�er dem Phantom Problem vermieden. Im Gegensatz zu HyperGraphDB werden logische �nderungsoperationen am verwalteten B-Baum beim Commit persistiert und sind somit dauerhaft gespeichert. 

% Die transaktionalen Eigenschaften von Apache Cassandra k�nnen in \cite nachvollzogen werden.
Ungeachtet dessen muss die Konsistenz innerhalb des Graphen sichergestellt werden: Sind Attributschl�ssel oder Kantenbezeichner als \texttt{UNIQUE} deklariert, d�rfen diese nicht von parallel ausgef�hrten Transaktionen ge�ndert werden. Titan implementiert hierf�r ein Sperrverfahren um den exklusiven Zugriff auf die entsprechenden Attribute bzw. Kanten zu garantieren. Beim Zugriff wird durch das GDBMS eine entsprechende Sperre gesetzt und bis zum Ende der Transaktion gehalten. Sperrkonflikte, welche durch das parallele Aktualisieren eindeutiger Attributschl�ssel oder Kantenbezeichner entstehen, z�hlen zu den dauerhaften Fehlern und m�ssen von der Anwendung behandelt werden.
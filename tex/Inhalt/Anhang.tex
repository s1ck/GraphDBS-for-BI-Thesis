\chapter{Anhang }
\label{cha:Anhang}

\section{GDBMS-Implementierungen}
\label{anh:vendor_list}

\renewcommand{\arraystretch}{1.25}
\begin{table}[h]
	\centering
	\begin{footnotesize}
   	\begin{tabular}{|m{2.5cm}|m{3.5cm}|>{\arraybackslash}m{9.5cm}|}
	\hline	
	\textbf{GDBMS} & \textbf{Hersteller} & \textbf{Website} \\
	\hline
   	Affinity 		& GoPivotal 			& \url{http://affinityng.cfapps.io/} \\
   	ArangoDB 		& triAGENS 				& \url{http://www.arangodb.org/} \\
   	Bitsy 			& Privatperson 			& \url{https://bitbucket.org/lambdazen/bitsy/} \\
   	DEX 			& Sparsity Technologies & \url{http://www.sparsity-technologies.com/dex} \\
   	Filament 		& Privatperson 			& \url{http://sourceforge.net/projects/filament/} \\
   	FlockDB 		& Twitter				& \url{https://github.com/twitter/flockdb} \\
   	GraphBase 		& FactNexus 			& \url{http://graphbase.net/} \\
   	GraphPack 		& Privatperson 			& \url{https://code.google.com/p/graphpack/} \\
   	G-Store 		& Forschungsprototyp	& \url{http://g-store.sourceforge.net/} \\
   	Horton 			& Microsoft 			& \url{http://research.microsoft.com/en-us/projects/ldg/} \\
   	HyperGraphDB 	& Kobrix Software		& \url{http://www.hypergraphdb.org/index} \\
   	InfiniteGraph 	& Objectivity			& \url{http://www.objectivity.com/infinitegraph} \\
   	Fallen-8 		& Privatperson 			& \url{http://www.fallen-8.com/} \\
   	Neo4j 			& Neo Technology 		& \url{http://www.neo4j.org/} \\
   	OQGRAPH 		& Open Query 			& \url{http://openquery.com/node/23} \\
   	OrientDB 		& Orient Technologies 	& \url{http://www.orientdb.org/} \\
   	RedisGraph 		& Privatperson 			& \url{https://github.com/tblobaum/redis-graph} \\
   	SGDB3 			& Forschungsprototyp 	& \url{http://ups.savba.sk/~marek/sgdb.html} \\
   	Titan 			& Aurelius 				& \url{http://thinkaurelius.github.io/titan/} \\
   	Trinity 		& Microsoft 			& \url{http://research.microsoft.com/en-us/projects/trinity/} \\
   	VertexDB 		& Privatperson 			& \url{https://github.com/stevedekorte/vertexdb} \\
   	\hline
   	\end{tabular} 
	\end{footnotesize}
	\setlength{\belowcaptionskip}{0.25cm}	
	\caption[GDBMS-Hersteller]{Liste der Webseiten untersuchter GDBMS.}
	\label{tab:anh_urls}
\end{table}
\renewcommand{\arraystretch}{1}

\section{Quellcode-Beispiele}

\subsection{Neo4j}

\subsubsection{CRUD-Operationen via Core API}
\label{anh:neo4j_native_api}

\lstset{language=Java, caption={Erzeugen eines Graphen unter Verwendung der nativen API.}, label=list:neo4j_native_api, escapeinside={(*@}{@*)}}
\begin{lstlisting}
// queue for the upload logs
channel.queue_declare(queue="upload_logs",
        durable=True)

// queue for the critical logs
channel.queue_declare(queue="critical_logs",
        durable=True)
 
// binding for all messages related to the p_upload component
channel.queue_bind(exchange="logging",
        queue="upload_logs",
        routing_key='*.p_upload')(*@ \label{src:bind1} @*)

// binding for all critical log messages
channel.queue_bind(exchange="logging",
        queue="critical_logs",
        routing_key='critical.*')(*@ \label{src:bind2} @*)
\end{lstlisting}

\subsubsection{Traversal Framework}
\label{anh:neo4j_traversal_framework}

\subsection{HypergraphDB}

\subsubsection{CRUD-Operationen via Java API}

\subsubsection{Traversierung via Java API}

\subsection{OrientDB}

\subsubsection{CRUD-Operationen via Blueprints API}
\label{anh:orientdb_blueprints_api}

\subsubsection{Traversierung via Java API}
\label{anh:orientdb_traverse_java}

\subsection*{Titan}

\subsubsection{CRUD-Operationen via Blueprints API}
\label{anh:titan_blueprints_api}

\section{Benchmark}

\subsection{Anfragen}
\label{anh:queries}


\chapter{Zusammenfassung und weitere Entwicklung}
\label{cha:Fazit}
\paragraph*{Dokumentation}

Neo4j
	- sehr gute Dokumentation
	- Weg der L�nge $k$ f�hrt zu $\mathcal{O}(\left|N(v_1)\right| \times \left|N(v_2)\right| \times \cdots \times \left|N(v_k)\right|)$
	- Support auf Mailingliste

HyperGraphDB
	- gut Dokumentation
	- Support auf Mailingliste
	
OrientDB
	- Dokumentation sehr eingeschr�nkt, inkonsistent, viel Quelltextrecherche
	- Support auf Mailingliste
	
Titan
	- gute Doku f�r junges Projekt
	- Support auf Mailingliste
	
\paragraph*{Datenmodell}

% Tabelle

- Neo4j: PGM + Knotenlabel
	- keine Objekteinbettung
	- keine Schemadefinition an Knoten und Kanten
- HyperGraphDB: 
	- Atom-Modell sehr generisch
	- Schema durch Klassendefinition
- OrientDB: 
	- Dokumentmodell als Basis
	- PGM auf Dokumente abgebildet 
	- Schema durch Klassendefinition
- Titan: PGM + TitanKey + TitanLabel
	- keine Schemadefinition an Knoten und Kanten
	- Einschr�nkung der Attribute
	- Einschr�nkung der Kanten
	- Einschr�nkung Datentypen
	- verschachtelte Attributwerte (Maps)
	- Vertex-centric Indices
	- unidirektionale Kanten

\paragraph*{Zugriffsmechanismen}

CRUD-Operationen:
	- Neo4j: CRUD
		- in nativer API und Cypher m�glich
		- Algorithmen unterst�tzten Einschr�nkungen auf Graphen (im Gegensatz zu Blueprints)
	- HyperGraphDB: CRUD
	- OrientDB: CRUD
	- Titan: CRUD
Traversierung:
	- Neo4j:
		- algorithmische Traversierung mittels Traversal Framework -> Java
	- HyperGraphDB
		- algorithmische Traversierung mittels Traversal Framework -> Java
	- OrientDB
		- TRAVERSE-Operator oder Gremlin
	- Titan:
		- Gremlin, turing-m�chtig
Erreichbarkeit
	- Neo4j:
		- native API stellt Algorithmen zur Verf�gung
		- Cypher bietet shortestPath allShortestPath-Funktionen an
	- HyperGraphDB
		- durch Traversierung umsetzbar
		- Dijkstra-Implementierung
	- OrientDB
		- durch Traversierung umsetzbar
		- shortestPath und dijkstra in API
	- Titan
		- durch Traversierung umsetzbar
Mustersuche
	- Neo4j:
		- Cypher erm�glicht die Definition beliebiger Mustergraphen
	- HyperGraphDB
	 	- keine native Unterst�tzung
		- einfache Muster durch entsprechende Pr�dikatkombinationen
	- OrientDB
		- keine native Unterst�tzung
		- evtl. durch Schachtelung von SELECT und TRAVERSE
	- Titan
		- Musterdefinition via Gremlin und table-Funktion
		- weniger elegant als Cypher (= komplexer)
Aggregation und Summierung
	- Neo4j:
		- Aggregation ja
		- Summierung nein
	- HyperGraphDB
		- Aggregation nein -> (hg.apply)
		- Summierung nein
	- OrientDB
		- Aggregation ja
		- Summierung nein
	- Titan
		- Aggregation ja
		- Summierung nein
Metriken
	- Neo4j: allShortestPaths -> grundlage f�r centrality ma�e
	- bis auf Kardinalit�t der Knoten- / Kantenmenge nichts zus�tzliches

Transaktionen
	- generell: Optimierung f�r Leseoperationen / Konflikte werden an Anwendung weitergegeben (OrientDB und Titan)
	- Neo4j: ACID, Locking, READ COMMITTED
	- HyperGraphDB: ACI(D), MVCC (GDBMS) + Locking (BerkeleyDB), SERIALIZABLE
	- OrientDB: ACID, MVCC, SERIALIZABLE
	- Titan: ACID, Locking (BerkeleyDB), SERIALIZABLE (BerkeleyDB)
	
\paragraph*{Speicherung und Caching}

- Neo4j: nativ
	- Trennung Topologie und Daten
	- Traversierung: O(1)
	- Fragmentierung der Stores
- HyperGraphDB: nicht-nativ
	- keine Trennung von Topologie und Daten
	- Speicherung in KV-Store 
	- Traversierung: O(logn)
- OrientDB: nativ
	- keine Trennung von Topologie und Daten 
	- Traversierung: O(N\_v)
	- Fragmentierung der Stores
	- attributierte Kanten sind zus�tzliche Indirektion\url{https://github.com/orientechnologies/orientdb/wiki/Performance-Tuning-Blueprints}
- Titan: nicht-nativ
	- keine Trennung von Topologie und Daten
	- Speicherung in KV-Store 
	- Traversierung O(logn)
	- Ein Vorteil gegen�ber OrientDB und HyperGraphDB, ist die M�glichkeit des direkten Zugriffes auf Attribute und inzidente Kanten eines Knotens unter Angabe der Knoten-Identit�t und der geforderten 
	Column.
	- Caching und langlaufende �nderungstransaktionen

- generell: Beschleunigung via Caches -> Hashtabellen  O(1) (wenn da)


OrientDB:
	- widerspr�chliche Dokumentation, wenig Beispiele
	- Bugs z.B. falsche Ergebnisse beim Traversieren, expand(shortestPath) und expand(dijkstra) haben 	kein Effekt

	- Ergebnisse zwischen Tiefen- und Breitensuche unterscheiden sich durch Verhindern des wiederholten Zugriffs

	- physische Repr�sentation: Deserialisieren der Attribute kann sich negativ auf Performance auswirken, attributierte Kanten ebenfalls

\paragraph*{Titan}

- Besonderheit: 
	- out-unique Einschr�nkung innerhalb der Instanz
	- in-unique entspricht Unique in Neo4j
	- verschachtelte Attributwerte
	- Vertex-centric Indices
	- unidirektionale Kanten

\begin{itemize}
	\item Traversieren durch Hubs (Knoten die mit n-1 Knoten verbunden sind) -> Performance?
\end{itemize}
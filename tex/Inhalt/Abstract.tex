\section*{Zusammenfassung}
\label{sec:Zusammenfassung}

In der vorliegenden Masterarbeit werden verschiedene Graphdatenbanksysteme in einer funktionalen und technischen Evaluation hinsichtlich ihrer Eignung f�r ein aktuelles Forschungsvorhaben der Abteilung Datenbanken der Universit�t Leipzig untersucht. Ziel des Forschungsprojektes ist die Integration von Unternehmensdaten in einem sog. Informationsnetzwerk und die darauf aufbauende graphenorientierte Analyse.

Im Rahmen der theoretischen Grundlagen wird zun�chst auf die erforderlichen graphentheoretischen Konzepte sowie auf Informationsnetzwerke und weitere Netzwerkarten eingegangen. Jede Netzwerkart ist dabei mit unterschiedlichen Einsatzgebieten verbunden, die sich daraus ergebenden Anforderungen f�hren zu einer Kategorisierung graphenbasierter Softwaresysteme in Graphdatenbanksysteme, Graph Processing Systems und Visualisierung- und Analysesoftware. Der Fokus wird auf Graphdatenbanksysteme gelegt, da diese sich durch ihre Ausrichtung auf lokale, traversierende Anfragen in Verbindung mit klassischen Datenbankfunktionalit�ten, wie zum Beispiel Mehrbenutzerf�higkeit und Konsistenzerhaltung, f�r das Forschungsprojekt eignen. Neben den verschiedenen Auspr�gungen von Graphdatenbanksystemen werden die eingesetzten Datenmodelle definiert. Die theoretische Vorbetrachtung schlie�t mit der Definition graphenspezifischer Operationen, welche f�r die Analyse von Informationsnetzwerken relevant sind.

Aufgrund der Vielzahl verschiedener Implementierungen werden im Rahmen der funktionalen Analyse zun�chst kategorisierte Anforderungen aus den Projektzielen abgeleitet und auf deren Grundlage die Systeme verglichen. Durch die Differenzierung in obligatorische und optionale Anforderungen konnten aus den urspr�nglich in Betracht gezogenen 22 Graphdatenbanksystemen vier Systeme ausgew�hlt werden: Neo4j, HyperGraphDB, OrientDB und Titan.\\
Die sich anschlie�ende, detaillierte Betrachtung der Systeme widmet sich dem Datenmodell, den Zugriffs- und Indexmechanismen, der Persistenz- und Cacheverwaltung sowie der Verteilung und Skalierbarkeit der einzelnen Systeme. Der Vergleich ergab, dass vor allem Neo4j und Titan aufgrund ihres hohen Funktionsumfangs an graphenspezifischen Operationen und durch ihre effizienten Speicher- und Verteilungsmechanismen f�r den Einsatz innerhalb des Forschungsprojektes in Frage kommen.\\
Die Leistungsf�higkeit beider Systeme wird in einem abschlie�enden Benchmark verglichen, dabei werden die Ausf�hrung analytischer Anfragen sowie die Anfrageformulierung in den jeweiligen Anfragesprachen bewertet. Der Vergleich ergab, dass sich auch in technischer Hinsicht beide Systeme aufgrund des gezeigten Leistungsverhaltens f�r das Projekt eignen, Neo4j jedoch pr�feriert wird, da es im Vergleich zu Titan mit Unterst�tzung von Cypher und Gremlin den gr��eren Funktionsumfang aufweist und im Benchmark bei der Mehrzahl der Anfragen ein besseres Antwortzeitverhalten zeigt.

%\newpage
%
%\section*{Abstract}
%\label{sec:Abstract}
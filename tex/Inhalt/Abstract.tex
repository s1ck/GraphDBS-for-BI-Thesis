\section*{Zusammenfassung}
\label{sec:Zusammenfassung}

In der vorliegenden Masterarbeit werden verschiedene Graphdatenbanksysteme in einer funktionalen und technischen Evaluation hinsichtlich ihrer Eignung f�r ein aktuelles Forschungsvorhaben der Abteilung Datenbanken der Universit�t Leipzig untersucht. Ziel des Forschungsprojektes ist die Integration von Unternehmensdaten in ein Informationsnetzwerk und eine darauf aufbauende graphenorientierte Analyse der Daten.

Im Rahmen der Auseinandersetzung mit den theoretischen Grundlagen der Thematik wird zun�chst auf die erforderlichen graphentheoretischen Konzepte sowie auf Informationsnetzwerke und weitere Netzwerkarten eingegangen. Jeder Netzwerktyp ist dabei mit unterschiedlichen Einsatzgebieten verbunden, die daraus resultierenden Anforderungen f�hren zur Einteilung der graphenbasierten Softwaresysteme in die Kategorien Graphdatenbanksysteme, Graph Processing Systems sowie Visualisierung- und Analysesoftware. Der Fokus wird auf Graphdatenbanksysteme gelegt, da diese sich durch ihre Ausrichtung auf lokale, traversierende Anfragen in Verbindung mit klassischen Datenbankfunktionalit�ten, wie zum Beispiel Mehrbenutzerf�higkeit und Konsistenzerhaltung, f�r das Forschungsprojekt eignen. Neben der Definition verschiedener Auspr�gungen von Graphdatenbanksystemen wird auf die eingesetzten Datenmodelle eingegangen. Die theoretische Vorbetrachtung schlie�t mit der Definition graphenspezifischer Operationen, die f�r die Analyse von Informationsnetzwerken relevant sind.

Die Vielzahl verschiedener Implementierungen macht es erforderlich, zun�chst im Rahmen der funktionalen Analyse kategorisierte Anforderungen aus den Projektzielen abzuleiten und auf deren Grundlage eine erste Auswahl zu treffen. Nach der Differenzierung in obligatorische und optionale Anforderungen konnten aus den urspr�nglich in Betracht gezogenen 22 Graphdatenbanksystemen vier Systeme ausgew�hlt werden: Neo4j, HyperGraphDB, OrientDB und Titan.\\
Die sich anschlie�ende detaillierte Untersuchung der Systeme betrachtet das Datenmodell, die Zugriffs- und Indexmechanismen, die Persistenz- und Cacheverwaltung sowie die Verteilung und Skalierbarkeit. Der Vergleich ergab, dass vor allem Neo4j und Titan aufgrund ihres hohen Funktionsumfangs an graphenspezifischen Operationen und durch ihre effizienten Speicher- und Verteilungsmechanismen f�r den Einsatz innerhalb des Forschungsprojektes in Frage kommen.\\
Die Leistungsf�higkeit beider Systeme wird in einem abschlie�enden Benchmark verglichen, dabei werden die Ausf�hrung analytischer Anfragen sowie die Anfrageformulierung in den jeweiligen Anfragesprachen bewertet. Die Messungen haben gezeigt, dass sich auch in technischer Hinsicht beide Systeme aufgrund des gezeigten Leistungsverhaltens f�r das Projekt eignen. Die Pr�ferenz liegt bei Neo4j, da es im Vergleich durch Unterst�tzung der deklarativen Anfragesprache Cypher und der imperativen Anfragesprache Gremlin den gr��eren Funktionsumfang aufweist und im Benchmark bei der Mehrzahl der Anfragen ein besseres Antwortzeitverhalten zeigt.
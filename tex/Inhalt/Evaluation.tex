\chapter{Evaluation von Graphdatenbanksystemen}
\label{cha:evaluation}

\section{Anforderungen im Rahmen aktueller Forschungsvorhaben}
\label{sec:anforderungen}

\section{Vorauswahl von Graphdatenbanksystemen}

\paragraph*{Nutzbarkeit und Zugang}

blubber blubber blub

\renewcommand{\arraystretch}{1.25}
\begin{table}[ht]
	\centering
	\begin{footnotesize}
   	\begin{tabular}{|m{2.25cm}|>{\centering}m{1.5cm}|>{\centering}m{2.5cm}|>{\centering}m{2.0cm}|c|c|>{\centering\arraybackslash}m{2cm}|}
	\hline
	\multicolumn{7}{|c|}{\textbf{Nutzbarkeit und Zugang}} \\
	\hline
   	GDBMS & Quell-\newline~offen & Dokumentation & Produktiv-\newline~system & Sprache* & Aktiv & Linux* \\   
   	\hline
   	Affinity		& \checkmark	& \checkmark	& \checkmark	& C++	& \checkmark 	& \checkmark \\
   	ArangoDB		& \checkmark	& \checkmark	& \checkmark	& C/C++	& \checkmark	& \checkmark \\	
   	Bitsy			& \checkmark	& \checkmark	& \checkmark	& Java	& \checkmark	& \checkmark \\
   	DEX				& - 			& \checkmark	& \checkmark	& C++	& \checkmark	& \checkmark \\
   	Filament		& \checkmark	& \checkmark	& -				& Java	& \checkmark	& \checkmark \\
   	FlockDB			& \checkmark	& (\checkmark)	& \checkmark	& Java	& \checkmark	& \checkmark \\
   	GraphBase		& -				& \checkmark	& \checkmark	& Java	& \checkmark	& \checkmark \\
   	GraphPack		& \checkmark	& -				& -				& Java	& -				& \checkmark \\
   	G-Store			& -				& \checkmark	& -				& C/C++	& -				& - \\
   	Horton			& -				& -				& k.A.\tablefootnote{keine Angabe}	& k.A.	& k.A.			& - \\
   	HypergraphDB	& \checkmark	& \checkmark	& \checkmark	& Java	& \checkmark	& \checkmark \\
   	InfiniteGraph	& -				& \checkmark	& \checkmark	& Java	& \checkmark	& \checkmark \\
   	Infogrid		& \checkmark	& \checkmark	& \checkmark	& Java	& \checkmark	& \checkmark \\
   	Fallen-8		& \checkmark	& -				& -				& C\#	& \checkmark	& \checkmark \\
   	Neo4j			& \checkmark	& \checkmark	& \checkmark	& Java	& \checkmark	& \checkmark \\
   	OQGRAPH			& \checkmark	& \checkmark	& \checkmark	& C		& \checkmark	& \checkmark \\
   	OrientDB		& \checkmark	& \checkmark	& \checkmark	& Java	& \checkmark	& \checkmark \\
   	RedisGraph		& \checkmark	& - 			& -				& Javascript & \checkmark	& \checkmark \\
   	SGDB3			& \checkmark	& -				& -				& Java	& -				& \checkmark \\
   	Titan			& \checkmark	& \checkmark	& \checkmark	& Java	& \checkmark	& \checkmark \\
   	Trinity			& -				& -				& k.A.			& k.A.	& k.A.			& - \\
   	VertexDB		& \checkmark	& \checkmark	& -				& C		& -				& \checkmark \\
   	\hline
   	\end{tabular} 
	\end{footnotesize}
	\setlength{\belowcaptionskip}{0.25cm}	
	\caption{Anforderungen an die Verwendbarkeit und Dokumentation von GDBMS. (*optional/informativ)}
	\label{tab:nutzung}
\end{table}
\renewcommand{\arraystretch}{1}

\paragraph*{Datenverwaltung und Datenmodellierung}

\renewcommand{\arraystretch}{1.25}
\begin{table}[ht]
	\centering
	\begin{footnotesize}
   	\begin{tabular}{|m{2.25cm}|>{\centering}m{3.5cm}|>{\centering}m{2.25cm}|>{\centering}m{1.5cm}|>{\centering}m{1.5cm}|>{\centering\arraybackslash}m{2.5cm}|}
	\hline
	\multicolumn{6}{|c|}{\textbf{Datenverwaltung und Datenmodellierung}} \\
	\hline
   	GDBMS & Datenmodell & Mehrere\newline~Datenbanken* & Schema* & ACID* & Integrit�ts-\newline bedingungen* \\   
   	\hline
   	Affinity		& Gerichteter, knotenattributierter Multigraph	& - & -	& \checkmark & \checkmark \\
   	ArangoDB		& PGM	& \checkmark & - & \checkmark & \checkmark \\
   	Bitsy			& PGM	& -	& -	& \checkmark & \checkmark \\
   	FlockDB			& Gerichteter, knotenattributierter, kantenbezeichneter Graph & \checkmark	& -	& -	& \checkmark \\
   	HypergraphDB	& PHGM	& -	& \checkmark & \checkmark & \checkmark \\
   	Infogrid		& Gerichteter, knotenattributierter, kantenbezeichneter Multigraph	& -	& \checkmark & \checkmark & \checkmark \\
   	Neo4j			& PGM	& -	& -	& \checkmark	& \checkmark \\
   	OQGRAPH			& Gerichteter, gewichteter Multigraph	& -	& -	& -	& \checkmark \\
   	OrientDB		& PGM	& -	& \checkmark	& \checkmark	& \checkmark \\
   	Titan			& PGM	& -	& \checkmark	& \checkmark	& \checkmark \\
   	\hline
   	\end{tabular} 
	\end{footnotesize}
	\setlength{\belowcaptionskip}{0.25cm}	
	\caption{Anforderungen hinsichtlich der Datenverwaltung und -modellierung innerhalb von GDBMS. (*optional/informativ)}
	\label{tab:verwaltung}
\end{table}
\renewcommand{\arraystretch}{1}

\paragraph*{Zugriff}

\renewcommand{\arraystretch}{1.25}
\begin{table}[ht]
	\centering
	\begin{footnotesize}
   	\begin{tabular}{|m{2.25cm}|>{\centering}m{1.65cm}|>{\centering}m{1.25cm}|>{\centering}m{1.25cm}|>{\centering}m{1.25cm}|>{\centering\arraybackslash}m{2cm}|>{\centering}m{1.5cm}|>{\centering\arraybackslash}m{1.5cm}|}
	\hline
	\multicolumn{8}{|c|}{\textbf{Zugriff}} \\
	\hline
   	GDBMS & Embedded\newline API & Remote \newline API* & Plugin \newline API* & CRUD & Traversierung & Anfrage-\newline sprache* & Bulk\newline Load \\   
   	\hline   
   	ArangoDB		& -				& \checkmark 	& - 			& \checkmark	& \checkmark & \checkmark	& \checkmark	\\
   	Bitsy			& \checkmark	& \checkmark	& -				& \checkmark 	& \checkmark & \checkmark	& -				\\
   	HypergraphDB	& \checkmark	& -				& - 			& \checkmark 	& \checkmark & - 			& -				\\
   	Neo4j			& \checkmark	& \checkmark	& \checkmark	& \checkmark	& \checkmark & \checkmark	& \checkmark	\\
   	OrientDB		& \checkmark	& \checkmark	& -				& \checkmark	& \checkmark & \checkmark	& \checkmark	\\
   	Titan			& \checkmark	& \checkmark	& \checkmark	& \checkmark	& \checkmark & \checkmark	& \checkmark	\\
   	\hline
   	\end{tabular} 
	\end{footnotesize}
	\setlength{\belowcaptionskip}{0.25cm}	
	\caption{Anforderungen an die Zugriffsm�glichkeiten von GDBMS. (*optional/informativ)}
	\label{tab:zugriff}
\end{table}
\renewcommand{\arraystretch}{1}

\paragraph*{Speicherung}

\renewcommand{\arraystretch}{1.25}
\begin{table}[ht]
	\centering
	\begin{footnotesize}
   	\begin{tabular}{|m{2.25cm}|>{\centering}m{2.5cm}|>{\centering}m{2.5cm}|>{\centering}m{2.5cm}|>{\centering\arraybackslash}m{2.5cm}|}
	\hline
	\multicolumn{5}{|c|}{\textbf{Speicherung}} \\
	\hline
   	GDBMS & Persistenz & Native Speicherung*  & In-Memory* & Index-\newline unterst�tzung \\
   	\hline   
   	Bitsy			& \checkmark	& -				& \checkmark	& \checkmark 	\\
   	HypergraphDB	& \checkmark	& \checkmark	& - 			& \checkmark 	\\
   	Neo4j			& \checkmark	& \checkmark	& -				& \checkmark	\\
   	OrientDB		& \checkmark	& \checkmark	& -				& \checkmark	\\
   	Titan			& \checkmark	& -				& -				& \checkmark	\\
   	\hline
	\end{tabular} 
	\end{footnotesize}
	\setlength{\belowcaptionskip}{0.25cm}	
	\caption{Anforderungen an die Datenspeicherung in GDBMS. (*optional/informativ)}
	\label{tab:speicherung}
\end{table}
\renewcommand{\arraystretch}{1}

\paragraph*{Skalierbarkeit und Verf�gbarkeit}

\renewcommand{\arraystretch}{1.25}
\begin{table}[ht]
	\centering
	\begin{footnotesize}
   	\begin{tabular}{|m{2.25cm}|>{\centering}m{2.5cm}|>{\centering}m{2.5cm}|>{\centering\arraybackslash}m{2.5cm}|}
	\hline
	\multicolumn{4}{|c|}{\textbf{Skalierbarkeit und Verf�gbarkeit}} \\
	\hline
   	GDBMS & Partitionierung* & Replikation*  & Backup* \\
   	\hline   
   	Bitsy			& -				& -				& \checkmark \\
   	HypergraphDB	& -				& \checkmark	& \checkmark \\
   	Neo4j			& -				& \checkmark	& \checkmark \\
   	OrientDB		& -				& \checkmark	& \checkmark \\
   	Titan			& \checkmark	& \checkmark	& \checkmark \\
   	\hline
	\end{tabular} 
	\end{footnotesize}
	\setlength{\belowcaptionskip}{0.25cm}	
	\caption{Anforderungen an die Skalierbarkeit und Verf�gbarkeit von GDBMS. (*optional/informativ)}
	\label{tab:speicherung}
\end{table}
\renewcommand{\arraystretch}{1}


Datenmodell - grunds�tzlich ist die Modellierung auch mit anderen Datenmodellen m�glich (z.B. Zwischenknoten), wird jedoch hier nicht ber�cksichtigt um die Menge der zu vergleichenden Systeme einzuschr�nken

\section{Neo4j}

Blubb

\subsection{Datenmodell}

Property-Graph + Labels (2.0)	

\subsection{Zugriffsm�glichkeiten}

Blubb

\subsection{Persistenz}

Blubb

\subsection{Verteilung}

Blubb

\section{HypergraphDB}

Blubb

\subsection{Datenmodell}

Blubb

\subsection{Zugriffsm�glichkeiten}

Blubb

\subsection{Persistenz}

Blubb

\subsection{Verteilung}

Blubb

\section{OrientDB}

Blubb

\subsection{Datenmodell}

Blubb

\subsection{Zugriffsm�glichkeiten}

Blubb

\subsection{Persistenz}

Blubb

\subsection{Verteilung}

Blubb

\section{Titan}

Blubb

\subsection{Datenmodell}

Blubb

\subsection{Zugriffsm�glichkeiten}

Blubb

\subsection{Persistenz}

Blubb

\subsection{Verteilung}

Blubb

\section{Bitsy}

Blubb

\subsection{Datenmodell}

Blubb

\subsection{Zugriffsm�glichkeiten}

Blubb

\subsection{Persistenz}

Blubb

\subsection{Verteilung}

Blubb

\section{Gegen�berstellung}

Blubb


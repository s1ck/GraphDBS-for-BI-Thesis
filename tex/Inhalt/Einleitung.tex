\chapter{Einleitung}
\label{cha:Einleitung}

\section{Motivation}
\label{sec:motivation}

\begin{itemize}
	\item Einfache Graphendefinition (vielleicht auch ein kurzer Abriss zur Geschichte der Graphentheorie? Eulers Br�ckenproblem)
	\item Bezug auf Anwendungen in denen die Beziehungen zwischen Objekten mindestens genauso wichtig sind wie die Objekte selbst
	\item kein GDBMS ist optimal f�r alle Anwendungen \cite{Shao:2012:MML:2213836.2213907}, da die Algorithmen von verschiedenen Datenstrukturen profitieren
	\item Anwendungsbeispiele f�r Graphen / Graphdatenbanken
	\item Graph als generische Datenstruktur f�r die Modellierung beliebiger Sachverhalte
	\begin{itemize}
		\item Wissens- und Informationsnetzwerke (Zitiernetzwerke, Semantic Web, \textbf{Unternehmensdaten}, ...)
		\item Soziale Netzwerke (Facebook, LinkedIn, ...)
		\item Technologische Netzwerke (Kommunikation, Verkehr, ...)
		\item Biologische Netzwerke (Biologie, Chemie, ...)
		\item Beispiele in \cite{Shao:2012:MML:2213836.2213907}
	\end{itemize}
	\item Fokus auf Unternehmensdaten. 
	\begin{itemize}
		\item Bisher: Datawarehouse + relationale DBMS
		\item Jetzt: Nutzen von Graphdatenbanksystemen zur erweiterten Analyse vernetzter Unternehmensdaten (Bezug auf BIIIG Paper / aktuelle Promotion)
		
	\end{itemize}
\end{itemize}

\begin{quote}
Compared with other database options, the  unique property of graph databases is the information encoded in the graph topology\cite{Ciglan:2012}.
\end{quote}

\section{Ziel der Arbeit}
\label{sec:ziel}

�bergeordnetes Ziel dieser Masterarbeit ist es verschiedene Graphdatenbanksysteme anhand definierter Anforderungen zu vergleichen, einige der Systeme auszuw�hlen und in einer Evaluation ihre Eignung f�r aktuelle Forschungsvorhaben in der Abteilung Datenbanken der Universit�t Leipzig zu untersuchen.

In diesem Zusammenhang sollen zun�chst graphenbasierte Softwaresysteme kategorisiert werden, der Fokus liegt hierbei auf Graphdatenbanksystemen und deren verschiedenen Auspr�gungen. Die theoretische Betrachtung existierender Datenmodelle, wie zum Beispiel des Property-Graph-Modells oder des Hypergraphs, wird durch die Definition einfacher und komplexer graphenspezifischer Operationen erg�nzt. In einer Vorauswahl sollen aktuelle Graphdatenbanksysteme zun�chst auf der Grundlage definierter Merkmale gegen�bergestellt werden. Einige dieser Merkmale, wie zum Beispiel Quelloffenheit, dauerhafte Speicherung und aktive Weiterentwicklung, werden als zwingend erforderlich angesehen. Hierdurch kann die Menge der zu untersuchenden Systeme reduziert und eine Auswahl getroffen werden. In der sich anschlie�enden Evaluation werden die ausgew�hlten Graphdatenbanksysteme detailliert betrachtet, der Fokus liegt auf Datenmodell und Zugriffsm�glichkeiten, sowie den verwendeten Datenstrukturen in Haupt- und Hintergrundspeicher. Neben der theoretischen Untersuchung funktionaler Eigenschaften sollen die Systeme in einem Benchmark gegen�bergestellt werden. Die Leistungsf�higkeit bei der Ausf�hrung grundlegender Operationen, wie der Manipulation des Graphen, ist dabei genauso immanent wie das Systemverhalten bei der Ausf�hrung komplexer Operationen. Zu diesen z�hlen unter anderem das Finden von Mustern innerhalb des Graphen oder das Berechnen von Pfaden mit definierten Einschr�nkungen. Abschlie�end sollen die St�rken, vor allem aber die Schw�chen der einzelnen Systeme aufgezeigt werden. Daraus l�sst sich eine eine Empfehlung f�r den weiteren Einsatz ableiten.

\section{Aufbau der Arbeit}

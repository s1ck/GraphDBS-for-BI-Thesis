\chapter{Graphdatenbanksysteme}

def siehe: \cite{DBLP:journals/corr/abs-1006-2361}

\section{Grundlegende Eigenschaften}

\begin{itemize}
	\item deklarativ vs. prozedural
	\item verteilt vs. zentral
	\item (haupt)speicher- vs. diskorientiert
	\item Isolationsstufen
	\item nativ vs. aufgesetzt
	\item lese- vs. schreiboptimiert
	\item batch-processing vs. realtime
	\item Indexunterst�tzung ja/nein
	\item Schema ja /nein
	\item Integrit�tsbedingungen ja/nein
	\item eingebetted vs. hosted
	\item graphenorientiertes Speicherformat ja/nein
	\item Einschr�nkungen bzgl. CAP Theorem
	\item Bulk-Import ja/nein
	\item Einzel- vs. Mehrnutzer
	\item Erweiterbarkeit ja/nein
	\item Integrit�tsbedingungen
\end{itemize}

\cite{Angles:2008}
\begin{itemize}
	\item Schema-Instanz-Konsistenz (Typ-Constraints an Knoten und Kanten. Wertebereiche f�r Attribute)
	\item Identit�t (Labels mit eindeutigen Namen)
	\item Referentielle Integrit�t
	\item Funktionale Abh�ngigkeiten
\end{itemize}

\section{Datenmodelle}

Ein Graphdatenbankmodell ist ein Modell, in welchem die Datenstrukturen f�r Schema und/oder Instanzen als direkter, eventuell benannter Graph modelliert sind. Datenmanipulation erfolgt durch graphenorientierte Operationen und Typkonstruktoren, Integrit�tsbedingungen k�nnen auf der Graphstruktur definiert werden. \cite{Angles:2008}

Ein durchg�ngiges Beispiel, was sowohl im relationalen, als auch in den jeweiligen Graphdatenmodellen beschrieben wird

\subsection{Property-Graph-Datenmodell}
\label{subsec:propgraph}

formale Definition: \cite{Ciglan:2012}

Schemaevolution

\subsection{Hypergraph-Datenmodell}

% auch Kombination mit Property-Graph m�glich

\subsection{Resource Description Framework}
\label{subsec:rdf}

\begin{itemize}
	\item spezieller Typ einer Graphdatenbank
	\item gerichteter, gelabelter Multigraph
	\item Fokus auf Inferenztechniken (und weniger auf Pfadsuchen)
	\item Properties und Beziehungen werden "gemischt"
	\item \url{http://www.quora.com/What-are-the-differences-between-a-Graph-database-and-a-Triple-store}
\end{itemize}

\section{Graphenspezifische Operationen}
\label{sec:operations}

\subsection{Grundlegende Operationen}

siehe \cite{Dominguez-Sal2011}, S. 31

\begin{itemize}
	\item CRUD von Knoten und Kanten (typbasiert (Schema), propertybasiert (Index))
	\item Nachbarschaftsanfrage (k-Nachbarn)
	\item Traversierung (BFS + DFS + Constraints)
	\item Aggregationen auf Basis von Properties (group by + min/max/avg/sum/count)
	\item Sortierung auf Basis von Properties
\end{itemize}

\subsection{Komplexe analytische Operationen}

\begin{itemize}
	\item Erreichbarkeit (k�rzeste Pfade, irgendein Pfad, mit und ohne Einschr�nkungen (z.B. Routing, Tourplanung))
	\item Zentralit�t von Knoten (Betweenness centrality)
	\item Musterbasierte Suche (exact vs. approximate)
	\item Flussalgorithmen
	\item Graph Summarization (siehe \cite{Zhao:2011:GCW:1989323.1989413})
	\item Drill-Down, Drill-Through, Slice \& Dice
	\item Statistische Anfragen (Knoten-/Kantenzahl, Zusammenhang, Knotengrade (Verteilung), Durchmesser...)
\end{itemize}
\section{Titan}

% quelloffen (Lizenz), Firma, Ver�ffentlichung, Version
% Sprache
- Java
% nativ / nicht nativ (Abbildung auf Key-Column-Store)
- Modellierung / Verarbeitung nativ
- Speicherung nicht-nativ 
- verschiedene Backends: Cassandra, HBase, BerkeleyDB
% embedded, remote
% zentral, verteilt (Replikation Partitionierung)
% disk-zentriert / in-memory

% Besonderheiten
	% Skalierbarkeit (Refernz auf Benchmark)

% Erl�uterungen / Version / Quellen
% Bezug auf BerkeleyDB
	% Vergleichbarkeit mit HyperGraphDB
	% Verteilung spielt im Benchmark keine Rolle

\subsection{Datenmodell}

% PGM (+eventuelle �nderungen), Knotenlabel?
% Knoten (Label?)
% Kanten (Richtung, Traversieren, Label, Semantik
% Properties (Typen, mehrfache Properties, Fehlen einer Property)
% Identit�t
% Integrit�t (modellinh�rent, Integrit�tsbedingungen)
% Anzahl der Datenbanken (wie Partitionierung)
% Schema
	% Flexibilit�t

\subsection{Zugriffsmechanismen, Transaktionen und Indexverwaltung}

% Varianten nennen (eingebettet, remote)
% Blueprints-Stack kurz erkl�ren

% CRUD-Operationen analog zu OrientDB (+Typdefinition)
	% Unterschiede zu OrientDB aufzeigen (z.B. Typen statt Strings)
	% beliebig durch API
	% angebotene Algorithmen (entsprechend OrientDB im Furnace Paket)
	% imperativ
% Gremlin als wesentlicher Inhalt
	% Paradigma (imperativ)
	% standardisiert (quasi-standard)
		% von allen GDBMS nutzbar, welche die Blueprints API implementieren
	% Definition
	% Manipulation (Einf�gen Knoten / Kante)
	% Traversierung
	% Mustersuche
	
\paragraph*{Transaktionen}

% obligatorisch
% Erzeugen (implizit, explizit)
% Schachtelung
% BerkeleyDB siehe OrientDB
	% Dauerhaftigkeit ist standardm��ig aktiviert (Quelltext)
	% REPEATABLE READ (keine Mehrbenutzeranomalien)
	% Behandlung von Deadlocks
% Transaktionen innerhalb von Titan?
% Anmerkung zu Cassandra und HBase

\paragraph*{Indexverwaltung}

% Primary Key
% Property-Index
	% Aktualisierung (autmatische, manuelle Indizes)
	% Beispiel
% Vertex-centric Index!
% Erweiterung?

\subsection{Persistenz- und Cacheverwaltung}

% BerkeleyDB: B-Baum, Transaktions-Log
% Abbildung des Graphen auf KV-Store (reicht!)
% Komplexit�t Nachbarschaft / Traversierung

\paragraph*{Cacheverwaltung}

% BerkeleyDB
% eigene Caches

\subsection{Verteilung und Skalierbarkeit}

% BerkeleyDB ausschlie�lich zentral
% Replikation und Partitionierung via Cassandra / HBase
% vielleicht kurz beschreiben (+Skalierbarkeit)